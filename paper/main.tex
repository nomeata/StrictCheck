% !!! PLEASE DON'T CHANGE THESE !!! INSTEAD DEFINE YOUR OWN texdirectives.tex !!!
% These flags are for the final camera ready version for POPL
% Draft, full version flags
\newif\ifdraft\drafttrue
\newif\iffull\fullfalse
\newif\ifanon\anonfalse
\newif\iflast\lasttrue

% !!! PLEASE DON'T CHANGE THESE !!! INSTEAD DEFINE YOUR OWN texdirectives.tex !!!
\makeatletter \@input{texdirectives} \makeatother

\documentclass[acmsmall,review]{acmart}\settopmatter{}

%% Note: Authors migrating a paper from PACMPL format to traditional
%% SIGPLAN proceedings format should change 'acmlarge' to
%% 'sigplan,10pt'.


%% Some recommended packages.
\usepackage{booktabs}   %% For formal tables:
                        %% http://ctan.org/pkg/booktabs
\usepackage{subcaption} %% For complex figures with subfigures/subcaptions
                        %% http://ctan.org/pkg/subcaption

\usepackage{xspace}
\usepackage{preamble}
\usepackage{relsize}
%\usepackage{mathpartir}

\makeatletter\if@ACM@journal\makeatother
%% Journal information (used by PACMPL format)
%% Supplied to authors by publisher for camera-ready submission
% \setcopyright{rightsretained}
% \acmJournal{PACMPL}
% \acmYear{2018} \copyrightyear{2018} \acmVolume{2} \acmNumber{POPL} \acmArticle{45} \acmMonth{1} \acmPrice{}\acmDOI{10.1145/3158133}

%% Copyright information
%% Supplied to authors (based on authors' rights management selection;
%% see authors.acm.org) by publisher for camera-ready submission
%\setcopyright{acmcopyright}
%\setcopyright{acmlicensed}
%\setcopyright{rightsretained}
%\copyrightyear{2017}           %% If different from \acmYear


%% Bibliography style
\bibliographystyle{ACM-Reference-Format}
%% Citation style
%% Note: author/year citations are required for papers published as an
%% issue of PACMPL.
\citestyle{acmauthoryear}   %% For author/year citations

\def\l{l}

\begin{document}

%% Title information
\title{Keep your Laziness in Check}
                                        %% when present, will be used in
                                        %% header instead of Full Title.
%\titlenote{with title note}             %% \titlenote is optional;
%                                        %% can be repeated if necessary;
%                                        %% contents suppressed with 'anonymous'
%\subtitle{}                     %% \subtitle is optional
%\subtitlenote{with subtitle note}       %% \subtitlenote is optional;
                                        %% can be repeated if necessary;
                                        %% contents suppressed with 'anonymous'


%% Author information
%% Contents and number of authors suppressed with 'anonymous'.
%% Each author should be introduced by \author, followed by
%% \authornote (optional), \orcid (optional), \affiliation, and
%% \email.
%% An author may have multiple affiliations and/or emails; repeat the
%% appropriate command.
%% Many elements are not rendered, but should be provided for metadata
%% extraction tools.

%% Author with single affiliation.
\author{Kenneth Fonner}
%\authornote{with author1 note}          %% \authornote is optional;
                                        %% can be repeated if necessary
%\orcid{nnnn-nnnn-nnnn-nnnn}             %% \orcid is optional
\affiliation{
%  \position{Position1}
%  \department{Department1}              %% \department is recommended
  \institution{University of Pennsylvania}            %% \institution is required
%  \streetaddress{Street1 Address1}
%  \city{City1}
%  \state{State1}
%  \postcode{Post-Code1}
  \country{USA}
}
\email{kwf@very.science}          %% \email is recommended

%% Author with single affiliation.
\author{Hengchu Zhang}
%\authornote{with author1 note}          %% \authornote is optional;
                                        %% can be repeated if necessary
%\orcid{nnnn-nnnn-nnnn-nnnn}             %% \orcid is optional
\affiliation{
%  \position{Position1}
%  \department{Department1}              %% \department is recommended
  \institution{University of Pennsylvania}            %% \institution is required
%  \streetaddress{Street1 Address1}
%  \city{City1}
%  \state{State1}
%  \postcode{Post-Code1}
  \country{USA}
}
\email{hengchu@seas.upenn.edu}          %% \email is recommended

%% Author with single affiliation.
\author{Leonidas Lampropoulos}
%\authornote{with author1 note}          %% \authornote is optional;
                                        %% can be repeated if necessary
%\orcid{nnnn-nnnn-nnnn-nnnn}             %% \orcid is optional
\affiliation{
%  \position{Position1}
%  \department{Department1}              %% \department is recommended
  \institution{University of Pennsylvania}            %% \institution is required
%  \streetaddress{Street1 Address1}
%  \city{City1}
%  \state{State1}
%  \postcode{Post-Code1}
  \country{USA}
}
\email{llamp@seas.upenn.edu}          %% \email is recommended

%% Paper note
%% The \thanks command may be used to create a "paper note" ---
%% similar to a title note or an author note, but not explicitly
%% associated with a particular element.  It will appear immediately
%% above the permission/copyright statement.
%\thanks{with paper note}                %% \thanks is optional
                                        %% can be repeated if necesary
                                        %% contents suppressed with 'anonymous'


%% Abstract
%% Note: \begin{abstract}...\end{abstract} environment must come
%% before \maketitle command
\begin{abstract}
StrictCheck.

Because LazyCheck was taken.
\end{abstract}


%% 2012 ACM Computing Classification System (CSS) concepts
%% Generate at 'http://dl.acm.org/ccs/ccs.cfm'.
\begin{CCSXML}
<ccs2012>
<concept>
<concept_id>10011007.10011006.10011008</concept_id>
<concept_desc>Software and its engineering~General programming languages</concept_desc>
<concept_significance>500</concept_significance>
</concept>
</ccs2012>
\end{CCSXML}

\ccsdesc[500]{Software and its engineering~General programming languages}
%% End of generated code


%% Keywords
%% comma separated list
\keywords{Random Testing, Laziness, Haskell}  %% \keywords is optional

%% \maketitle
%% Note: \maketitle command must come after title commands, author
%% commands, abstract environment, Computing Classification System
%% environment and commands, and keywords command.
\maketitle

% \lk{
% \begin{description}
%   \item[slant] \the\fontdimen1\font
%   \item[interword space] \the\fontdimen2\font
%   \item[interword stretch] \the\fontdimen3\font
%   \item[interword shrink] \the\fontdimen4\font
%   \item[extra space] \the\fontdimen7\font
%   \item[xspaceskip] \the\xspaceskip
%   \item[hyphenchar] \the\hyphenchar\font
% \end{description}
% }

\section{Introduction}
\label{sec:intro}

\begin{itemize}
\item Laziness is a fundamental feature in Haskell \leo{WhyFPmatters?}
\item However, reasoning about programs with laziness (performance, behavior) is hard.
\item It is very easy to use a too-lazy or too-strict implementation that yields different results.
\item FRP quote here?
\item
\item Example: folds.
\item \lk{fold} to calculate sum breaks for large lists because of thunks
\item \lk{fold'} is good for some, but not for average because of whnf.
\item We need an implementation that is strict in the pair as well.
\item
\item Which leads to the natural question, how can we test the laziness behavior of functions?
\item Even worse, how can we even {\em specify} the laziness behavior?
\end{itemize}

Our technical contributions are as follows:
\begin{itemize}
\item We introduce a novel approach to specifying the laziness behavior of functions that is
{\em precise} \leo{define}. \leo{Also add drawbacks/partiality here? If not here, where in the intro?}
\item We propose a random testing approach to test such specifications.
\item We implement StrictCheck. \leo{In order for these last two points to be different, we need
something about the cool/awesome hacks in the implementation to point out here}
\item As a side contribution? We fix the function generaton of QuickCheck to produce lazy/partial functions instead
of fully strict ones all the time.
\end{itemize}
%
Section~\ref{sec:related} discusses related work. We conclude and draw
directions for future work in Section~\ref{sec:concl}.

\section{Strictness Specifications, by Example}
\label{sec:quickchick}

\leo{Punned names for sections? Or is that just an urns thing? :P }

\begin{itemize}
\item In this section we will showcase our proposed specification language \leo{It's not a language tho}
\item while introducing the running example of our paper: maps.
\item Maps are particularly well-suited for a discussion of laziness because of the interesting different
possible behaviors: strict-maps, value-lazy maps \leo{Point out Data.Map.Strict vs Data.Map}, and fully lazy maps
\item For simplicity, we will use binary search trees as maps, assuming the keys are integers and forego balancing
as the added complexity is irrelevant to the laziness discussion. \leo{Can we claim it wouldn't matter at all? I think so}
\end{itemize}

\paragraph*{Binary Search Trees}

\leo{data structures - code - short explanation}

\paragraph*{Characterizing Laziness}

\begin{itemize}
\item Specification of laziness - TreeD data structure \leo{Mention how it will be generically derived}
\item Strict / Spine-strict maps
\item Fully lazy map as a between spec
\end{itemize}

\section{Evaluation}
\label{sec:eval}

\section{Related Work}
\label{sec:related}

\leo{If we want to break up this into smaller chunks, use paragraph*}

\section{Conclusion and future work}
\label{sec:concl}

Machine Learning for automatically synthesizing specs? Let's go crazy!

%% Acknowledgments
\begin{acks}                            %% acks environment is optional
                                        %% contents suppressed with 'anonymous'
  %% Commands \grantsponsor{<sponsorID>}{<name>}{<url>} and
  %% \grantnum[<url>]{<sponsorID>}{<number>} should be used to
  %% acknowledge financial support and will be used by metadata
  %% extraction tools.
We are grateful to
%
Matt Parsons,
Matthew Weaver,
Jennifer Paykin,
Antal Spector-Zabusky,
and the Penn PLClub 
for their useful comments. 
This material is based upon work supported by the
\grantsponsor{GS100000001}{National Science
  Foundation}{http://dx.doi.org/10.13039/100000001} under Grant
No.~\grantnum{GS100000001}{1421243} ({\em Random Testing for Language
Design}), Grant No.~\grantnum{GS100000001}{1521523} ({\em Expeditions
in Computing: The Science of Deep Specification}), and \leo{Kenny,
Hengchu, ask for grant numbers names} Any opinions, findings, and
conclusions or recommendations expressed in this material are those of
the author and do not necessarily reflect the views of the National
Science Foundation.

\end{acks}

%% Bibliography
%\bibliography{bibfile}
\bibliography{local}


%% Appendix
%\appendix
%\section{Appendix}
% 
%Text of appendix \ldots

\end{document}
